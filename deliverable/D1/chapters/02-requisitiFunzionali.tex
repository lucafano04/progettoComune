\chapter{Requisiti Funzionali} 
    \section{Requisiti funzionali comuni a tutti gli utenti}
        \begin{rfList}
            \rfItem \textbf{Visualizzazione città} Il sistema deve permettere a tutti gli utenti di poter visualizzare gli attributi demografici e riguardanti la soddisfazione della città. A fianco degli attributi sarà inoltre 	presente la mappa con focus sulla città divisa per zone colorate in base al relativo grado di soddisfazione media e i relativi pulsanti per modificarne le impostazioni, nel caso in 	cui l'utente fosse autenticato come utente analista sarà possibile sostituire alla mappa una tabella contenente le zone nelle quali è divisa la città.
            \rfItem \textbf{Interazione con la mappa} Il sistema deve permettere a tutti gli utenti di poter muovere, interagire e modificare la visualizzazione della mappa. In particolare deve essere possibile modificare il focus centrale della mappa trascinando il cursore, deve essere possibile modificare lo zoom attraverso la rotella del mouse oppure attraverso i pulsanti presenti nell'angolo della mappa, deve essere possibile interagire con le varie zone cliccando sulle stesse e infine deve essere possibile, quando si è all'interno della "visualizzazione città", modificare la tipologia di zona con la quale si può interagire sulla mappa oppure, nel caso in cui si avesse i permessi da analista, si può cambiare la visualizzazione da mappa a tabella.
            \rfItem \textbf{Visualizzazione zona} Il sistema deve permettere a tutti gli utenti di poter visualizzare gli attributi, demografici e riguardanti la soddisfazione, oltre ai servizi forniti della zona selezionata (circoscrizione o quartiere). A fianco degli attributi sarà inoltre presente la mappa, con focus sulla zona di selezione, divisa per zone colorate in base al relativo grado di soddisfazione e al focus centrale della mappa, saranno inoltre visualizzati ai vari angoli della mappa i relativi pulsanti per modificarne le impostazioni.
            \rfItem \textbf{Elenco strutture} Il sistema deve permettere a tutti gli utenti di visualizzare, per il servizio selezionato, una più dettagliata descrizione di tutte le strutture, presenti all'interno dell'area di interesse, che erogano tale servizio. Tale visualizzazione mostrerà una tabella numerata con all'interno il nominativo delle varie strutture e affianco una mappa con contrassegnato la posizione delle strutture presenti nella tabella.
            \rfItem \textbf{Multi lingua} Il sistema deve permettere a tutti gli utenti di poter modificare la lingua nella quale vengono presentati i testi. Le lingue presenti per la selezioni sono: Italiano, Inglese e Tedesco. Attraverso il menù a tendina presente nella header sarà possibile selezionare la lingua di preferenza, infine successivamente alla selezione la pagina verrà ricaricata nella lingua selezionata.
        \end{rfList}
    \section{Requisiti funzionaliper gli utenti non loggati}
        \begin{rfList}
            \rfItem \textbf{Login} Il sistema deve permettere a tutti gli utenti non loggati di accedere, se presente, al loro account. Tale funzionalità sarà accessibile premendo il tasto di login presente nella header il quale reindirizzerà alla pagina del service provider della provincia di Trento dalla quale sarà infine possibile accedere tramite servizi Single Sing On (SSO). Successivamente al processo di autenticazione l'utente verrà reindirizzato alla "visualizzazione città" e verrà sostituita l'icona del login con l'icona corrispondente a quella del profilo dal quale si è fatto l'accesso.
        \end{rfList}
    \section{Requisiti funzionali per tutti gli utenti loggati}
        \begin{rfList}
            \rfItem \textbf{Logout} Il sistema deve permettere a tutti gli utenti loggati di potersi scollegare dall'account al quale sono attualmente collegati, riportando così l'utente allo stato di utente non loggato e reindirizzandolo alla "visualizzazione città". Sarà possibile eseguire il logout attraverso il menù a tendina presente nella header.
        \end{rfList}     
    \section{Requisiti funzionali per i sondaggisti}
        \begin{rfList}
            \rfItem \textbf{Visualizzazione sondaggi} Il sistema deve permettere agli utenti sondaggisti di poter visualizzare in sezioni distinte le liste di sondaggi e le interfacce per l'aggiunta di sondaggi. In particolare il sistema deve presentare in due liste distinte i sondaggi non ancora caricati a sistema e quelli caricati a sistema, inoltre a fianco delle due liste sarà presente l'interfaccia per creare o caricare nuovi sondaggi.
            \rfItem \textbf{Gestione sondaggi} Il sistema deve permettere agli utenti sondaggisti di poter aggiungere, continuare, eliminare, salvare e completare i sondaggi non ancora caricati a sistema. In particolare deve essere possibile aggiungere un sondaggio creandone uno nuovo oppure caricandone uno, deve essere possibile continuare a modificare un sondaggio selezionandone uno dall'apposita visualizzazione sondaggi e infine deve essere possibile eliminare, salvare e inviare un sondaggio, con tutti i voti annessi ad esso, premendo gli appositi pulsanti presenti all'interno dell'interfaccia.
            \rfItem \textbf{Visualizzazione voti} Il sistema deve permettere agli utenti sondaggisti di poter visualizzare in sezioni distinte i dati relativi ai voti già inseriti all'interno del sondaggio in coro, le interfacce per la gestione dei voti di sondaggi e le interfacce per la gestione del sondaggio. In particolare il sistema deve presentare una sezione contenente le statistiche parziali generali e quelle relative ai vari quartieri, deve presentare la lista contenente i voti precedenti e le interfacce per gestire i voti e il sondaggio.
            \rfItem \textbf{Gestione voti} Il sistema deve permettere agli utenti sondaggisti di poter aggiungere o rimuovere i voti ai sondaggi in sospeso, ciò sarà possibile attraverso due apposite interfacce. Per aggiungere i voti sarà necessario inserire il quartiere di residenza del cittadino e a scelta volontaria dello stesso la propria fascia d'età, premendo il pulsante apposito il sistema caricherà dunque l'interfaccia necessaria per il voto, completando e inviando il voto il processo di aggiunta voto sarà dunque finito. Per eliminare i voti basterà invece premere il pulsante apposito sul voto presente nella apposita lista.
        \end{rfList}
    \section{Requisiti funzionali per gli analisti}
        \begin{rfList}
            \rfItem \textbf{Interazione con la tabella} Il sistema deve permettere agli utenti analisti di poter muovere, interagire e modificare la visualizzazione della tabella. In particolare deve essere possibile modificare il focus principale della tabella attraverso la rotella del mouse oppure attraverso la barra presente a lato della tabella, deve essere possibile interagire con le varie zone cliccando sul nome delle stesse e deve infine essere possibile, quando si è all'interno della visualizzazione della città, cambiare la visualizzazione da tabella a mappa.
            \rfItem \textbf{Accesso completo agli attributi} Il sistema deve permettere agli utenti analisti di poter avere accesso ad un maggior numero di attributi e ad una categorizzazione di essi in base all'area tematica degli stessi. In particolare deve essere possibile, ogni tal volta che un'utente analista si trova all'interno della "visualizzazione zona", visualizzare tutte le diverse categorie di attributi relativi a tale zona. Deve infine essere possibile selezionare la categoria della quale si vuole visualizzare gli attributi permettendo una visualizzazione settoriale e specifica del quartiere in questione.
            \rfItem \textbf{Analisi attraverso storici} Il sistema deve permettere agli utenti analisti di poter visualizzare per qualsiasi attributo uno storico. In particolare, attraverso la visualizzazione tramite categorie sarà possibile visualizzare uno storico relativo ad ogni attributo presente nella categoria specifica, ovvero sarà possibile visualizzabile al fianco di ogni attributo un grafico filtrabile per data di acquisizione del dato.
        \end{rfList}
    \section{Requisiti funzionali per gli amministratori}
        \begin{rfList}
            \rfItem \textbf{Approvazione-Disapprovazione dati sondaggisti} Il sistema deve permettere agli amministratori di visualizzare tramite una pagina dedicata, accessibile dalla top-bar dopo aver effettuato il login, una tabella contenente il riassunto dei dati relativi ai sondaggi inseriti dai sondaggisti con relativo stato di approvazione, inoltre saranno disponibili dei pulsati per visualizzare nel dettaglio i dati inseriti dai sondaggisti, per approvare i dati inseriti dai sondaggisti, per richiedere la modifica dei dati inseriti dai sondaggisti e per rifiutare i dati inseriti dai sondaggisti.
            \rfItem \textbf{Modifica dati statici} Il sistema deve permettere agli amministratori di visualizzare tramite una pagina dedicata, accessibile dalla top-bar dopo aver effettuato il login, una pagina di modifica dei dati relativi ai servizi e/o altro da inserire manualmente nel sistema, dalla stessa pagina sarà possibile aggiungere, modificare e/o eliminare i dati inseriti manualmente.
            \rfItem \textbf{Modifica utenti abilitati} Il sistema deve permettere agli amministratori di visualizzare un riepilogo, tramite pagina dedicata accessibile solo dagli amministratori, degli utenti registrati nel sistema, con la possibilità di visualizzare i dettagli di un utente, di modificare i dati di un utente, di eliminare un utente e di visualizzare i ruoli di un utente, inoltre deve essere possibile assegnare e/o rimuovere un ruolo ad un utente e abilitare un nuovo utente nel sistema. L'abilitazione di un utente nel sistema comporterà l'invio di una mail all'utente notificando l'avvenuta abilitazione.
            \rfItem \textbf{Visualizzazione richieste e risposte} Il sistema deve permettere agli amministratori di visualizzare le richieste inviate dalle circoscrizioni al comune e di mandare risposte ad esse. Inoltre il sistema deve permettere di visualizzare le risposte inviate tramite una tabella accessibile dalla top-bar dopo aver effettuato il login, nella quale verranno visualizzati l'oggetto e il testo della richiesta e se presente, l'oggetto e il testo della risposta del comune.
        \end{rfList}
    \section{Requisiti funzionali per le circoscrizioni}
        \begin{rfList}
            \rfItem \textbf{Aggiunta/modifica dati circoscrizioni} Il sistema deve permettere alle circoscrizioni di aggiungere e/o modificare i dati relativi ai servizi e/o altro da inserire nel sistema, tramite una pagina dedicata accessibile dalla top-bar dopo aver effettuato il login.
            \rfItem \textbf{Invio richiesta al comune} Il sistema deve permettere alle circoscrizioni l'invio di richieste al comune tramite una pagina dedicata accessibile dalla top-bar dopo aver effettuato il login, nella quale verrà inserito l'oggetto della richiesta e il testo della richiesta. Una volta inviata la richiesta verrà inviata una mail al comune e verrà visualizzato un messaggio di conferma dell'invio della richiesta.
            \rfItem \textbf{Visualizzazione richieste e risposte} Il sistema deve permettere alle circoscrizioni di visualizzare le richieste inviate al comune e le risposte ricevute da esso tramite una tabella, nella quale verranno visualizzati l'oggetto e il testo della richiesta e se presenti l'oggetto e il testo delle varie risposte.
        \end{rfList}