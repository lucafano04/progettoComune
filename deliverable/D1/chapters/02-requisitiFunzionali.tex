\chapter{Requisiti Funzionali} 
    \section{Requisiti funzionali comuni a tutti gli utenti}
        \begin{rfList}
            \rfItem \textbf{Homepage} Il sistema deve consentire a tutti gli utenti di essere in grado di visualizzare integralmente la mappa del comune di Trento divisa per quartieri non appena si apre la web-app.
            \rfItem \textbf{Interazione con la mappa} Il sistema deve permettere a tutti gli utenti di interagire con la mappa del comune di Trento, in particolare deve essere possibile: ingrandire, rimpicciolire e spostarsi all'interno della mappa tramite il mouse o i pulsanti.
            \rfItem \textbf{Multi lingua} Il sistema deve contenere dei pulsanti a forma di bandiera per cambiare lingua su tutte le pagine del sistema, le lingue che devono essere disponibili nel sistema saranno: italiano, tedesco, e inglese. Quando si preme su uno di questi pulsanti la pagina si ricaricherà nella lingua selezionata e il pulsante della lingua precedentemente selezionata tornerà ad essere disponibile.
            \rfItem \textbf{Accesso dati quartieri} Il sistema deve permettere a tutti gli utenti di selezionare qualunque dei vari quartieri della città. Selezionare un quartiere consentirà all'utente di visualizzare i dati generici (num popolazione, soddisfazione, età media, servizi,\dots) relativi al quartiere selezionato. Inoltre la mappa visualizzata sposterà il focus e si ingrandirà su di questo. Quando un quartiere è selezionato verrà evidenziato, sarà inoltre possibile de-selezionarlo cliccando nuovamente sullo stesso quartiere o cliccando su di un altro quartiere.
            \rfItem \textbf{Accesso dati specifici quartieri} I dati dei vari quartieri saranno visualizzabili in modo più esteso quando viene cliccato su uno specifico dato relativo ad un quartiere (se disponibile). Questo comporterà la visualizzazione di una tabella coi dati estesi relativi al quartiere selezionato, sarà possibile chiudere la tabella cliccando su un pulsante di chiusura. 
        \end{rfList} 
    \section{Requisiti funzionali per tutti gli utenti loggati}
        \begin{rfList}
            \rfItem \textbf{Autenticazione} Il sistema deve permettere a tutti gli utenti loggati di accedere al loro account premendo un tasto di login in altro a destra, il quale renderizzerà gli utenti alla pagina del service provider della provincia di Trento al quale accederanno tramite: il Sistema Pubblico di Identità Digitale (SPID), la Carta Nazionale dei Servizi (CNS), la Carta di Identità Elettronica (CIE) o la carta Provinciale dei Servizi (CPS). Nel caso in cui l'utente non avesse un ruolo abilitato questo verrà reindirizzato alla pagina principale con un messaggio di errore tramite pop-up che informerà dei mancati permessi per accedere al sistema.
            \rfItem \textbf{Cambio icona login} Successivamente al processo di autenticazione per qualsiasi utente loggato verrà sostituita l'icona del login con l'icona corrispondente al tipo di profilo con il quale si è fatto l'accesso.
        \end{rfList}     
    \section{Requisiti funzionali per i sondaggisti}
        \begin{rfList}
            \rfItem \textbf{Visualizzazione dati sondaggisti} Il sistema deve permettere ai sondaggisti di visualizzare tramite una pagina dedicata l'interfaccia per gestire i sondaggi. In questa interfaccia deve essere presente una tabella contenente il riassunto dei dati relativi ai sondaggi inserite da loro stessi con relativo stato di approvazione e nel caso in cui non fossero ancora stati approvati saranno inoltre disponibili anche dei pulsanti per eliminare e/o modificare i dati inseriti.
            \rfItem \textbf{Accesso come sondaggista} Il sistema successivamente al processo di autenticazione reindirizzerà automaticamente i vari account sondaggisti alla corrispettiva pagina dedicata, così facendo verrà velocizzata e semplificata la procedura d'accesso limitando inoltre le funzionalità fornite ai sondaggisti.
            \rfItem \textbf{Creazione nuovi sondaggi} Il sistema deve permettera ai sondaggisti di creare nuovi sondaggi. Questo deve essere possibile in due maniere diverse: creando un nuovo sondaggio vuoto oppure caricando un file contenente i dati di un sondaggio in corso. Questi sondaggi "in sospeso" potranno essere cancellati oppure presentati agli amministratori per essere accettati nel sistema o rifiutati. I dati inseriti dei sondaggi in sospeso non saranno visibili agli altri utenti e non saranno neanche considerati dal sistema fino a quando non verranno approvati da un utente amministratore.
            \rfItem \textbf{Svolgimento sondaggi} Il sistema deve permettere ai sondaggisti di aggiungere o rimuovere i voti ai sondaggi in sospeso. Tramite una pagina apposita, sarà possibile inserire o eliminare nuovi voti, aggiungendo informazioni sulla circoscrizione di appartenenza e a scelta volontaria del cittadino la propria età.
        \end{rfList}
    \section{Requisiti funzionali per analisti, circoscrizioni e amministratori}
        \begin{rfList}
            \rfItem \textbf{Modifica Homepage} Per gli utenti: analista, circoscrizione e amministratore, all'interno della Homepage verrà fornita la possibilità di visualizzare una tabella contenente le informazioni più importanti dei vari quartieri (Nome, percentuale di soddisfazione, numero di abitanti) al posto della mappa, sarà inoltre possibile passare dalla visualizzazione della mappa a quella della tabella (o viceversa) attraverso l'utilizzo di un pulsante di selezione.
            \rfItem \textbf{Accesso come analista, circoscrizione o amministratore} Successivamente al processo di autenticazione per gli utenti: analista, circoscrizione e amministratore, verranno reindirizzati alla Homepage modificata con la visualizzazione della città sotto forma di tabella.
        \end{rfList}
    \section{Requisiti funzionali per gli analisti}
        \begin{rfList}
            \rfItem \textbf{Visualizzazione Analisti Database} Il sistema deve permettere agli analisti la possibilità di visualizzare tutti i dati presenti nella base di dati, questo tramite apposite tabelle accessibili. Le tabelle saranno accessibili tramite link resi disponibili nella top-bar dopo che l'utente ha eseguito il login. Per ogni tabella, riguardante i dati, l'analista potrà decidere di filtrare e/o orinare la tabella per uno o più campi, alcuni campi potrebbero non essere filtrabili e/o ordinabili. La tabella visualizzata sarà suddivisa in pagine con un numero di righe predefinite ma modificabile.
            \rfItem \textbf{Visualizzazione Analisti Storico} Il sistema deve permettere agli analisti di visualizzare degli storici riassuntivi tramite grafici dinamici i quali potranno essere filtrati per data di acquisizione del dato visualizzato. Il sistema permetterà inoltre di confrontare due quartieri selezionati dall'analista, mostrando l'andamento del dato selezionato nel tempo, tramite due serie sullo stesso grafico. Il sistema permetterà inoltre di confrontare due dati diversi di uno stesso quartiere, mostrando l'andamento dei due dati nel tempo, tramite due serie sullo stesso grafico. 
        \end{rfList}
    \section{Requisiti funzionali per gli amministratori}
        \begin{rfList}
            \rfItem \textbf{Approvazione-Disapprovazione dati sondaggisti} Il sistema deve permettere agli amministratori di visualizzare tramite una pagina dedicata, accessibile dalla top-bar dopo aver effettuato il login, una tabella contenente il riassunto dei dati relativi ai sondaggi inseriti dai sondaggisti con relativo stato di approvazione, inoltre saranno disponibili dei pulsati per visualizzare nel dettaglio i dati inseriti dai sondaggisti, per approvare i dati inseriti dai sondaggisti, per richiedere la modifica dei dati inseriti dai sondaggisti e per eliminare i dati inseriti dai sondaggisti.
            \rfItem \textbf{Modifica dati statici} Il sistema deve permettere agli amministratori di visualizzare tramite una pagina dedicata, accessibile dalla top-bar dopo aver effettuato il login, una pagina di modifica dei dati relativi ai servizi e/o altro da inserire manualmente nel sistema, dalla stessa pagina sarà possibile aggiungere, modificare e/o eliminare i dati inseriti manualmente.
            \rfItem \textbf{Modifica utenti abilitati} Il sistema deve permettere agli amministratori di visualizzare un riepilogo, tramite pagina dedicata accessibile solo dagli amministratori, degli utenti registrati nel sistema, con la possibilità di visualizzare i dettagli di un utente, di modificare i dati di un utente, di eliminare un utente e di visualizzare i ruoli di un utente, inoltre deve essere possibile assegnare e/o rimuovere un ruolo ad un utente e abilitare un nuovo utente nel sistema. L'abilitazione di un utente nel sistema comporterà l'invio di una mail all'utente notificando l'avvenuta abilitazione.
            \rfItem \textbf{Visualizzazione richieste e risposte} Il sistema deve permettere agli amministratori di visualizzare le richieste inviate dalle circoscrizioni al comune e le risposte inviate tramite una tabella accessibile dalla top-bar dopo aver effettuato il login, nella quale verranno visualizzati l'oggetto della richiesta, il testo della richiesta, se è presente la risposta del comune, l'oggetto della risposta e il testo della risposta.
        \end{rfList}
    \section{Requisiti funzionali per le circoscrizioni}
        \begin{rfList}
            \rfItem \textbf{Aggiunta/modifica dati circoscrizioni} Il sistema deve permettere alle circoscrizioni di aggiungere e/o modificare i dati relativi ai servizi e/o altro da inserire nel sistema, tramite una pagina dedicata accessibile dalla top-bar dopo aver effettuato il login.
            \rfItem \textbf{Invio richiesta al comune} Il sistema deve permettere alle circoscrizioni l'invio di richieste al comune tramite una pagina dedicata accessibile dalla top-bar dopo aver effettuato il login, nella quale verrà inserito l'oggetto della richiesta e il testo della richiesta. Una volta inviata la richiesta verrà inviata una mail al comune e verrà visualizzato un messaggio di conferma dell'invio della richiesta.
            \rfItem \textbf{Visualizzazione richieste e risposte} Il sistema deve permettere alle circoscrizioni di visualizzare le richieste inviate al comune e le risposte ricevute dal comune tramite una tabella, nella quale verranno visualizzati l'oggetto della richiesta, il testo della richiesta, se è presente la risposta del comune, l'oggetto della risposta e il testo della risposta.
        \end{rfList}