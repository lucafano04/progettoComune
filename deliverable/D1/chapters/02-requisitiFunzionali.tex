\chapter{Requisiti Funzionali} 
\thispagestyle{stdPage}
    \section{Requisiti funzionali comuni a tutti gli utenti}
        \begin{rfList}
            \rfItem \textbf{Homepage} Il sistema deve consentire a tutti gli utenti di essere in grado di visualizzare la mappa del comune di Trento divisa per quartieri non appena si apre la web-app.
            \rfItem \textbf{Multi lingua} Il sistema deve contenere dei pulsanti a forma di bandiera per cambiare lingua su tutte le pagine del sistema le lingue che devono essere disponibili nel sistema saranno selezionando tra italiano, tedesco, e inglese. Quando si preme su uno di questi pulsanti la pagina si ricaricherà nella lingua selezionata e il pulsante della lingua precedentemente selezionata tornerà ad essere disponibile.
            \rfItem \textbf{Accesso dati quartieri} Il sistema deve permettere a tutti gli utenti di selezionare qualunque dei vari quartieri della città. Selezionare un quartiere consentirà all'utente di visualizzare i dati generici (num popolazione, felicità, età media, servizi) relativi al quartiere selezionato, inoltre la mappa visualizzata si sposterà sul quartiere selezionato e si ingrandirà. Apparirà dunque un pop-up sulla mappa con i dati prima citati. Quando un quartiere è selezionato verrà evidenziato, sarà inoltre possibile de-selezionare il quartiere selezionato.
            \rfItem \textbf{Accesso dati specifici quartieri} I dati dei vari quartieri saranno visualizzabili in modo più esteso quando viene cliccato su un dato relativo ad un quartiere. Questo comporterà la "apparizione" di una tabella contenente nel dettaglio tutti i dati relativi al dato generale del quartiere selezionato. Sarà possibile chiudere la tabella cliccando su un pulsante di chiusura, oppure cliccando su un altro dato generale del quartiere, oppure deselezionando il quartiere.
        \end{rfList} 
    \section{Requisiti funzionali per tutti gli utenti loggati}
        \begin{rfList}
            \rfItem \textbf{Autenticazione} Il sistema deve permettere a tutti gli utenti loggati di accedere al loro account premendo un tasto di login in altro a destra, il quale renderizzerà gli utenti alla pagina del service provider della provincia di Trento al quale accederanno tramite: il Sistema Pubblico di Identità Digitale (SPID), la Carta Nazionale dei Servizi (CNS), la Carta di Identità Elettronica (CIE), la carta Provinciale dei Servizi (CPS). Una volta eseguito l'accesso si verrà renderizzati alla pagina personale del sistema ed ogni utente potrà accedere alla sua dashboard personale basata sul ruolo abilitato per l'utente, se l'utente non ha un ruolo abilitato non potrà allora verrà reindirizzato alla pagina principale del sistema con un messaggio di errore tramite un pop-up che informerà l'utente che non ha i permessi per accedere al sistema. Se l'utente è già loggato al posto del tasto di login comparirà un menù a tendina con le opzioni: profilo, logout.
        \end{rfList}
    \section{Requisiti funzionali per gli analisti}
        \begin{rfList}
            \rfItem \textbf{Visualizzazione Analisi} Il sistema deve permettere agli analisti la possibilità di visualizzare tutti i dati presenti nella base di dati, tramite apposite tabelle accessibili tramite link che saranno resi disponibili nella top-bar dopo che l'utente ha eseguito il login. Ogni tabella di visualizzazione dei dati l'analista potrà decidere di filtrare e/o orinare la tabella per uno o più campi, alcuni campi potrebbero non essere filtrabili e/o ordinabili. La tabella visualizzata sarà suddivisa in pagine con un numero di righe predefinite ma modificabile.
            \rfItem \textbf{Visualizzazione Analisi Storico} Il sistema deve permettere agli analisti di visualizzare degli storici riassuntivi tramite grafici dinamici i quali potranno essere filtrati per data di acquisizione del dato visualizzato. Il sistema permetterà inoltre di confrontare due quartieri selezionati dall'analista, mostrando l'andamento del dato selezionato nel tempo, tramite due serie sullo stesso grafico. Il sistema permetterà inoltre di confrontare due dati diversi di uno stesso quartiere, mostrando l'andamento dei due dati nel tempo, tramite due serie sullo stesso grafico. 
        \end{rfList}
    \section{Requisiti funzionali per i sondaggisti}
        \begin{rfList}
            \rfItem \textbf{Visualizzazione dati sondaggisti} Il sistema deve permettere ai sondaggisti di visualizzare tramite una pagina dedicata, accessibile dalla top-bar dopo aver effettuato il login, una tabella contenente il riassunto dei dati relativi ai sondaggi inseriti da loro stessi con relativo stato di approvazione e se non approvati dei pulsanti per eliminare e/o modificare i dati inseriti. 
            \rfItem \textbf{Caricamento dati sondaggisti} Il sistema deve permettera ai sondaggisti di aggiungere i dati relativi ai sondaggi eseguiti sul territorio comunale tramite una pagina apposita nella quale verranno inseriti i dati relativi al riepilogo del sondaggio quali il numero di voti cumulato e raggruppati per quartiere. I dati inseriti dai sondaggisti non saranno visibili agli altri utenti e non saranno neanche considerati dal sistema fino a quando non verranno approvati da un utente amministratore.
            \rfItem \textbf{Modifica ed eliminazione dati sondaggisti} Il sistema deve permettere ai sondaggisti di modificare e/o eliminare i dati inseriti da loro stessi andando nella apposita pagina di visualizzazione dei sondaggi inseriti da loro stessi e selezionando il sondaggio che si vuole modificare e/o eliminare. 
        \end{rfList}
    \section{Requisiti funzionali per gli amministratori}
        \begin{rfList}
            \rfItem \textbf{Approvazione-Disapprovazione dati sondaggisti } Il sistema deve permettere agli amministratori di visualizzare tramite una pagina dedicata, accessibile dalla top-bar dopo aver effettuato il login, una tabella contenente il riassunto dei dati relativi ai sondaggi inseriti dai sondaggisti con relativo stato di approvazione, inoltre saranno disponibili dei pulsati per visualizzare nel dettaglio i dati inseriti dai sondaggisti, per approvare i dati inseriti dai sondaggisti, per richiedere la modifica dei dati inseriti dai sondaggisti e per eliminare i dati inseriti dai sondaggisti.
            \rfItem \textbf{Modifica dati statici} Il sistema deve permettere agli amministratori di visualizzare tramite una pagina dedicata, accessibile dalla top-bar dopo aver effettuato il login, una pagina di modifica dei dati relativi ai servizi e/o altro da inserire manualmente nel sistema, dalla stessa pagina sarà possibile aggiungere, modificare e/o eliminare i dati inseriti manualmente.
            \rfItem \textbf{Modifica utenti abilitati} Il sistema deve permettere agli amministratori di visualizzare un riepilogo, tramite pagina dedicata accessibile solo dagli amministratori, degli utenti registrati nel sistema, con la possibilità di visualizzare i dettagli di un utente, di modificare i dati di un utente, di eliminare un utente e di visualizzare i ruoli di un utente, inoltre deve essere possibile assegnare e/o rimuovere un ruolo ad un utente e abilitare un nuovo utente nel sistema. L'abilitazione di un utente nel sistema comporterà l'invio di una mail all'utente notificando l'avvenuta abilitazione.
        \end{rfList}