\chapter{Requisiti Funzionali} 
    \section{Requisiti funzionali comuni a tutti gli utenti}
        \begin{rfList}
            \rfItem \textbf{Visualizzazione città} Il sistema deve permettere a tutti gli utenti di visualizzare gli attributi demografici e riguardanti la soddisfazione della città. A fianco degli attributi sarà inoltre presente la mappa, con focus sulla città divisa per zone colorate in base al relativo grado di soddisfazione media, e i relativi pulsanti per modificarne le impostazioni. Nel caso in cui l'utente fosse autenticato come utente analista sarà inoltre possibile sostituire alla mappa una tabella contenente le zone nelle quali è divisa la città e viceversa.
            \rfItem \textbf{Interazione con la mappa} Il sistema deve permettere a tutti gli utenti di muovere, interagire e modificare la visualizzazione della mappa. In particolare deve essere possibile modificare il focus centrale e lo zoom della mappa utilizzando le funzionalità offerte dal fornitore del servizio (OpenStreetMap), deve essere possibile interagire con le varie zone cliccando su di esse. Inoltre deve essere possibile, quando si è all'interno della "visualizzazione città", modificare la tipologia di zona con la quale si può interagire sulla mappa. Infine, nel caso in cui si avesse i permessi da analista, deve essere possibile cambiare la visualizzazione da mappa a tabella.
            \rfItem \textbf{Visualizzazione zona} Il sistema deve permettere a tutti gli utenti di visualizzare gli attributi demografici, gli attributi riguardanti la soddisfazione e gli attributi riguardanti i servizi forniti della zona selezionata (circoscrizione o quartiere). A fianco degli attributi sarà inoltre presente la mappa, con focus sulla zona di selezione, divisa per zone colorate in base al relativo grado di soddisfazione. Saranno inoltre visualizzati ai vari angoli della mappa i relativi pulsanti per modificarne le impostazioni.
            \rfItem \textbf{Elenco strutture} Il sistema deve permettere a tutti gli utenti di visualizzare, per il servizio selezionato, una più dettagliata descrizione di tutte le strutture, presenti all'interno dell'area di interesse, che erogano tale servizio. Tale visualizzazione mostrerà una tabella numerata con all'interno il nominativo delle varie strutture e affianco una mappa con contrassegnate le posizioni delle strutture presenti nella tabella.
            \rfItem \textbf{Multi lingua} Il sistema deve permettere a tutti gli utenti di modificare la lingua nella quale vengono presentati i testi. Le lingue presenti per la selezioni sono: Italiano, Inglese e Tedesco. Sarà possibile selezionare la lingua di preferenza attraverso il menù a tendina presente nella header, successivamente alla selezione la pagina verrà ricaricata nella lingua selezionata.
        \end{rfList}
    \section{Requisiti funzionali per gli utenti non loggati}
        \begin{rfList}
            \rfItem \textbf{Login} Il sistema deve permettere a tutti gli utenti non loggati di autenticarsi al sistema per accedere ai privilegi forniti al loro ruolo. Tale funzionalità sarà accessibile premendo il tasto di login presente nella header il quale reindirizzerà alla pagina del service provider della provincia di Trento dalla quale sarà infine possibile accedere tramite servizi Single Sing On (SSO). Successivamente al processo di autenticazione l'utente verrà reindirizzato alla "visualizzazione città" e verrà sostituita l'icona del login con l'icona corrispondente a quella del profilo dal quale si è fatto l'accesso.
        \end{rfList}
    \section{Requisiti funzionali per tutti gli utenti loggati}
        \begin{rfList}
            \rfItem \textbf{Logout} Il sistema deve permettere a tutti gli utenti loggati di scollegarsi dall'account al quale sono attualmente collegati, riportando così l'utente allo stato di utente non loggato e reindirizzandolo alla "visualizzazione città". Sarà possibile eseguire il logout attraverso il menù a tendina presente nella header.
        \end{rfList}     
    \section{Requisiti funzionali per i sondaggisti}
        \begin{rfList}
            \rfItem \textbf{Visualizzazione sondaggi} Il sistema deve permettere agli utenti sondaggisti di visualizzare in sezioni distinte le liste dei propri sondaggi e le interfacce per l'aggiunta di sondaggi. In particolare il sistema deve presentare in due liste distinte i sondaggi non ancora caricati a sistema e quelli caricati a sistema, inoltre a fianco delle due liste sarà presente l'interfaccia per creare o caricare nuovi sondaggi.
            \rfItem \textbf{Gestione sondaggi} Il sistema deve permettere agli utenti sondaggisti di aggiungere, continuare, eliminare, salvare e completare i sondaggi non ancora caricati a sistema. In particolare deve essere possibile aggiungere un sondaggio creandone uno nuovo oppure caricandone uno, deve essere possibile continuare a modificare un sondaggio selezionandone uno dall'apposita visualizzazione sondaggi e infine deve essere possibile eliminare, salvare e inviare un sondaggio, con tutti i voti annessi ad esso, premendo gli appositi pulsanti presenti all'interno dell'interfaccia.
            \rfItem \textbf{Visualizzazione voti} Il sistema deve permettere agli utenti sondaggisti di visualizzare in sezioni distinte i dati relativi ai voti già inseriti all'interno di un sondaggio in corso, le interfacce per la gestione dei voti di sondaggi e le interfacce per la gestione del sondaggio. In particolare il sistema deve presentare una sezione contenente le statistiche parziali generali e quelle relative ai vari quartieri, deve presentare la lista contenente i voti precedenti e le interfacce per gestire i voti e il sondaggio.
            \rfItem \textbf{Gestione voti} Il sistema deve permettere agli utenti sondaggisti di aggiungere o rimuovere i voti dai sondaggi in sospeso, ciò sarà possibile attraverso due apposite interfacce. Per aggiungere i voti sarà necessario inserire il quartiere di residenza del cittadino e, a scelta volontaria dello stesso, la sua fascia d'età. Premendo il pulsante apposito il sistema caricherà dunque l'interfaccia necessaria per il voto, completando e inviando il voto il processo di aggiunta voto sarà dunque finito. Per eliminare i voti basterà invece premere il pulsante apposito sul voto presente nella apposita lista.
        \end{rfList}
    \section{Requisiti funzionali per gli analisti}
        \begin{rfList}
            \rfItem \textbf{Interazione con la tabella} Il sistema deve permettere agli utenti analisti di muovere, interagire e modificare la visualizzazione della tabella. In particolare deve essere possibile modificare il focus principale della tabella attraverso la rotella del mouse oppure attraverso la barra presente a lato della tabella, deve essere possibile interagire con le varie zone cliccando sul nome delle stesse e deve infine essere possibile, quando si è all'interno della visualizzazione della città, cambiare la visualizzazione da tabella a mappa.
            \rfItem \textbf{Accesso completo agli attributi} Il sistema deve permettere agli utenti analisti di avere accesso ad un maggior numero di attributi e ad una categorizzazione di essi in base all'area tematica degli stessi. In particolare deve essere possibile, ogni volta che un utente analista si trova all'interno della "visualizzazione zona", visualizzare tutte le diverse categorie di attributi relativi a tale zona. Deve infine essere possibile selezionare la categoria della quale si vuole visualizzare gli attributi permettendo una visualizzazione settoriale e specifica del quartiere in questione.
            \rfItem \textbf{Analisi attraverso storici} Il sistema deve permettere agli utenti analisti di visualizzare per qualsiasi attributo uno storico. In particolare, attraverso la visualizzazione tramite categorie sarà possibile visualizzare uno storico relativo ad ogni attributo presente nella categoria specifica, ovvero sarà possibile visualizzabile al fianco di ogni attributo un grafico filtrabile per data di acquisizione del dato.
        \end{rfList}
    \section{Requisiti funzionali per le circoscrizioni}
        \begin{rfList}
            \rfItem \textbf{Visualizzazione richieste} Il sistema deve permettere agli utenti circoscrizione di visualizzare in sezioni distinte una lista di richieste e l'interfaccia per l'invio di nuove richieste. In particolare il sistema deve presentare la lista delle richieste inviate con annesso il loro stato di successo e, se presente, il testo della relativa risposta. A fianco della lista deve essere presente l'interfaccia per creare una nuova richiesta per gli amministratori.
            \rfItem \textbf{Gestione richieste} Il sistema deve permettere agli utenti circoscrizione di aggiungere, visualizzare ed eliminare le richieste destinate agli amministratori. In particolare deve essere possibile: aggiungere richieste per gli amministratori attraverso un'interfaccia apposita che permetterà l'inserimento di un titolo ed un corpo della richiesta oltre al semplice pulsante di invio agli amministratori, visualizzare una specifica richiesta già inviata selezionandola all'interno dell'elenco ed infine deve essere possibile eliminare le richieste alle quali non è ancora stata data risposta.
            \rfItem \textbf{Gestione ruoli Circoscrizione} Il sistema deve permettere agli utenti circoscrizione di assegnare o rimuovere i ruoli agli utenti. In particolare deve essere possibile assegnare i ruoli di: sondaggista e analista della circoscrizione. Tali utenti avranno accesso alle funzionalità fornite dal ruolo al quale sono stati assegnati con maggiori limitazione sulle funzionalità dei ruoli. Deve inoltre essere possibile, per l'utente circoscrizione, rimuovere il ruolo di sondaggista o analista della circoscrizione a patto che l'utente circoscrizione faccia parte della stessa circoscrizione alla quale l'utente al quale si vuole rimuovere il ruolo è stato assegnato.
            \rfItem \textbf{Modifica informazioni servizi Circoscrizione} Il sistema deve permettere agli utenti circoscrizione di aggiungere o rimuovere informazioni riguardanti i servizi forniti ai cittadini presenti all'interno della circoscrizione e le relative strutture che li forniscono. In particolare deve essere possibile aggiungere una struttura, appartenente alla circoscrizione di appartenenza, che fornisce un servizio, inserendo il nome dell'associazione e nel caso in cui fosse fornita anche la posizione sulla mappa. Deve inoltre essere possibile modificare o rimuovere le informazioni riguardanti i servizi già presenti a sistema facenti parte della circoscrizione di appartenenza.
        \end{rfList}
    \section{Requisiti funzionali per gli amministratori}
        \begin{rfList}
            \rfItem \textbf{Valutazione sondaggi} Il sistema deve permettere agli utenti amministratore di visualizzare, approvare o rifiutare i sondaggi inviati dai sondaggisti. In particolare deve essere possibile accedere all'interfaccia per la valutazione, successivamente si avrà una lista di sondaggi che non si ha ancora valutato, sarà quindi possibile selezionando un sondaggio e valutare tale sondaggio. Nel caso in cui la valutazione fosse negativa sarà necessario inserire una motivazione del rifiuto.
            \rfItem \textbf{Gestione richieste delle circoscrizioni} Il sistema deve permettere agli utenti amministratore di visualizzare e rispondere alle richieste delle circoscrizioni. In particolare deve essere possibile visualizzare una lista delle richieste alle quali non è ancora stata data risposta, ordinata in base alla data di arrivo, sarà dunque possibile selezionare le varie richieste, visualizzarle e inviare una risposta.
            \rfItem \textbf{Gestione ruoli Amministratori} Il sistema deve permettere agli utenti amministratore di assegnare o rimuovere i ruoli agli utenti. In particolare deve essere possibile assegnare il ruolo di: sondaggista, analista o circoscrizione. Tali utenti avranno accesso a tutte le funzionalità fornite dal ruolo al quale sono stati assegnati. Deve inoltre essere possibile, per l'utente amministratore, rimuovere il ruolo di sondaggista, analista o circoscrizione.
            \rfItem \textbf{Modifica informazioni servizi} Il sistema deve permettere agli utenti amministratore di aggiungere o rimuovere informazioni riguardanti i servizi forniti al cittadino e le strutture che li forniscono. In particolare deve essere possibile aggiungere una struttura che fornisce un servizio inserendo il nome dell'associazione e, nel caso in cui fosse fornita, anche la posizione sulla mappa. Deve inoltre essere possibile modificare o rimuovere le informazioni riguardanti i servizi già presenti a sistema.
        \end{rfList}