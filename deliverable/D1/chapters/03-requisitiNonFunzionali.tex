\chapter{Requisiti Non Funzionali} 
    \begin{rnfList}
        \rnfItem \textbf{Compatibilità} La web app deve risultare compatibile con le seguenti versioni di browser: Chrome 80+, Firefox 80+, Safari 14+, Edge 80+ fornendo a ciascuna una pari esperienza per quanto riguarda il numero delle funzionalità disponibili.
        \rnfItem \textbf{Velocità di risposta} Il sistema deve essere in grado di fornire i dati richiesti all'utente entro 2 secondi dalla loro richiesta, ciò per garantire una esperienza ottimale verso l'utente.
        \rnfItem \textbf{Multi Utenza} Il sistema deve riuscire a gestire almeno 100 utenti connessi simultaneamente, senza che nessuna funzionalità venga compromessa o rallentata, in questo modo l'accesso alla web app sarà garantito sia agli utenti che devono operare sui dati e soa agli utenti che devono usufruire delle informazioni fornite dalla web app.
        \rnfItem \textbf{Sicurezza Dati} I dati dovranno essere memorizzati assicurandosi che solo gli utenti loggati abbiano la possibilità di accedere al database direttamente, il programma si assicurerà quindi che gli utenti che non possiedono i requisiti appositi non possano accedere tramite backend a dati a loro non accessibili. Le varie transazioni verranno inoltre protette tramite protocollo \texttt{HTTPS}.
        \rnfItem \textbf{Backup Plan} Il sistema dovrà prevedere un piano di backup dei dati in modo sicuro tramite connessione di trasferimento \texttt{SFTP} su un server locale interno all'edificio (6h/12h/1d) e un backup meno frequente remoto (1d/3d/1w). In questo modo nel caso in cui il server fallisse sarà possibile riassestare il sistema tramite i backup principali, nel caso in cui avvenisse invece un disastro naturale o un tentativo di compromissione dei dati sarà possibile fare affidamento al backup remoto.
        \rnfItem \textbf{Capacità di caricamento} Il sistema deve permettere ai Sondaggisti di caricare i sondaggi a sistema in meno di 5 minuti. Questo per garantire che i dati possano essere caricati in tempi ragionevoli e che i sondaggisti possano continuare a lavorare senza interruzioni.
        \rnfItem \textbf{Aggiornamento Dati} Il sistema deve permettere di aggiungere/modificare/eliminare i dati regolarmente mantenendo una struttura logica intatta. In questo modo gli analisti potranno esaminare i dati in modo corretto, senza errori e soprattutto avere una visione chiara e precisa dei dati. Inoltre vantaggio di questa funzionalità è che i dati saranno sempre aggiornati e quindi anche gli utenti non loggati potranno vedere i dati più recenti disponibili.
        \rnfItem \textbf{Aggiornamento dati ricavati} I dati del sistema che sono ricavati devono poter essere aggiornati automaticamente, attraverso dei collegamenti a database esterni esistenti, in caso di variazioni ai dati di questi ultimi, il sistema rifletterà le modifiche.
        \rnfItem \textbf{Invecchiamento Dati} I dati sulla soddisfazione caricati hanno validità massima di circa 6 mesi al fine di mantenerli attuali. In questo modo si evita che i dati siano obsoleti e che le decisioni prese siano basate su dati non più validi.
        \rnfItem \textbf{Facilità d'uso} La parte grafica generale deve essere di facile utilizzo per tutti gli utenti, la parte della web-app disponibile a tutti gli utenti deve essere comprensibile fin dal primo utilizzo ed entro 10 minuti dovrebbe essere chiaro a chiunque come funzioni l'app nella sua interezza. Per gli utenti loggati si richiederà di seguire una lezione di non più di 1 ora per imparare ad utilizzare tutte le funzionalità del sistema.
        \rnfItem \textbf{Facilità di Navigazione} Il sistema farà uso di un design accessibile tramite una navigazione con top-bar accessibile sia da schermi con risoluzione desktop, laptop e tablet. La parte di aggiungi/modifica dati, aggiungi/modifica utenti abilitati sono esentate da questa regola in quanto sono funzionalità riservate agli amministratori al quale accederanno principalmente da schermi desktop/laptop. Il resto dell'applicazione dovrà essere accessibile anche da dispositivi mobili.
        \rnfItem \textbf{Multilingua} La lingua selezionata inizialmente deve essere l'italiano, però devono essere disponibili anche le lingue: inglese e tedesco. Questo per garantire che gli utenti stranieri possano utilizzare l'applicazione senza problemi.
    \end{rnfList}