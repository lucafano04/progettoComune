\chapter{Implementazione}
L'applicazione è stata sviluppata usando il linguaggio di programmazione ``\texttt{TypeScript}'' sia per la parte di \textit{frontend} che per quella di \textit{backend}. Per la parte di \textit{backend} è stato sfruttato il \textit{runtime system} ``\texttt{Node.js}'' e il \textit{framework} ``\texttt{Express.js}'' per la creazione e gestione del \textit{server}. Per la parte di \textit{frontend} è stato utilizzato il \textit{framework} ``\texttt{Vue.js}'' con la libreria ``\texttt{PrimeVue}'' per alcune componenti grafiche. Il \textit{database} utilizzato è ``\texttt{MongoDB}'' e per la gestione delle dipendenze è stato utilizzato ``\texttt{npm}''.\newline
Si è inoltre scelto di usare \texttt{Vite} come \textit{bundler} per la parte di \textit{frontend} e \texttt{Webpack} per la parte di \textit{backend}. Oltre a questo per alcuni stili della libreria \texttt{PrimeVue} è stato usato \texttt{Tailwind CSS} e come conseguenza è stato usato \texttt{PostCSS} per la gestione dei fogli di stile. \newline
La scelta di usare \texttt{TypeScript} è scaturita dalla necessità di avere un controllo maggiore sul \textit{type-checking} e per avere una maggiore manutenibilità del codice, è sata creata infatti una vera e propria gerarchia di tipi per la gestione dei dati sia lato \textit{frontend} che lato \textit{backend}. \newline
La scelta del presente \textit{stack} tecnologico è stata fatta in base al materiale fornito dal corso ed conoscenze pregresse di alcuni membri del gruppo.

\section{Repository Organization}
    Il codice del progetto, disponibile presso la seguente repository \url{https://github.com/lucafano04/progettoComune}, è stato organizzato seguendo la seguente struttura:
    \dirtree{%
        .1 /.
        .2 /.github.
        .3 /workflows\DTcomment{Directory per le action di GitHub (Compilazione \LaTeX{} e test)}.
        .2 /.vscode\DTcomment{Directory per le impostazioni di Visual Studio Code}.
        .2 /APIdoc.
        .3 api.yaml\DTcomment{Documentazione \texttt{API}}.
        .2 /app\DTcomment{Directory per il codice di \textit{backend}}.
        .3 /db.
        .4 /models\DTcomment{Directory per i modelli del \textit{database}}.
        .4 index.ts\DTcomment{File di inizializzazione del \textit{database}}.
        .4 schemas.ts\DTcomment{File per la definizione degli schemi del \textit{database}}.
        .3 /routes\DTcomment{Directory per le rotte e gli \textit{endpoints} delle \texttt{API}}.
        .3 /tests\DTcomment{Directory per i test \texttt{Jest} per il \textit{backend}}.
        .3 /utils\DTcomment{Directory per le \textit{utility} di \textit{backend}}.
        .3 app.ts\DTcomment{File di inizializzazione dell'applicazione \texttt{Express.js}}.
        .3 variables.ts\DTcomment{File per la definizione delle variabili globali e ambientali}.
        .2 /deliverable\DTcomment{Directory per i \textit{deliverable} \LaTeX{}}.
        .3 /D*\DTcomment{Directory per il \textit{deliverable} D*, con * numero del \textit{deliverable}}.
        .3 /images\DTcomment{Directory comune per le immagini di tutti i \textit{deliverable}}.
        .2 /src\DTcomment{Directory per il codice di \textit{frontend}}.
        .3 /assets\DTcomment{Directory per gli \textit{asset} dell'applicazione}.
        .3 /components\DTcomment{Directory per le componenti dell'applicazione}.
        .3 /utils\DTcomment{Directory per le \textit{utility} di \textit{frontend}}.
        .3 App.vue\DTcomment{Componente radice dell'applicazione}.
        .3 index.css\DTcomment{File per il foglio di stile globale}.
        .3 main.ts\DTcomment{File di inizializzazione dell'applicazione \texttt{Vue.js}}.
        .2 /types.
        .3 /Circoscrizioni\DTcomment{Tipi per le circoscrizioni}.
        .3 /Dati\DTcomment{Tipi per i dati comuni a circoscrizioni e quartieri}.
        .3 /Quartieri\DTcomment{Tipi per i quartieri}.
        .3 /Sondaggi\DTcomment{Tipi per i sondaggi}.
        .3 /Utenti\DTcomment{Tipi per gli utenti}.
        .3 /Voti\DTcomment{Tipi per i voti}.
        .3 index.d.ts\DTcomment{File per il raggruppamento dei tipi}.
        .2 .env.example\DTcomment{File di esempio per le variabili ambientali}.
        .2 .gitignore\DTcomment{File per la definizione dei file da ignorare}.
        .2 index.html\DTcomment{Pagina HTML di base dell'applicazione}.
        .2 index.ts\DTcomment{File di inizializzazione dell'applicazione}.
        .2 package.json\DTcomment{File per la definizione delle dipendenze}.
        .2 pitch.pptx\DTcomment{Presentazione del progetto}.
        .2 *.config.js\DTcomment{File per la configurazione di \texttt{Webpack}, \texttt{TailWind} e \texttt{PostCSS}. Sostituendo * con la configurazione desiderata}.
        .2 tsconfig.*.json\DTcomment{File vari per la configurazione di \texttt{TypeScript}. Sostituendo * con la configurazione desiderata}.
        .2 vite.config.ts\DTcomment{File per la configurazione di \texttt{Vite} per la parte di \textit{frontend}}.
    }
\section{\textit{Branching Strategy} e organizzazione del lavoro}
    Per la gestione del lavoro si è scelto di utilizzare la piattaforma \texttt{GitHub} e di adottare una strategia di \textit{branching} basata su \textit{GitFlow}. In particolare si è deciso di utilizzare i seguenti \textit{branch}:
    \begin{description}
        \item[\texttt{main}] \textit{Branch} principale, contiene il codice stabile e funzionante;
        \item[\texttt{frontend}] \textit{Branch} per lo sviluppo della parte di \textit{frontend};
        \item[\texttt{MongoDB-Backend}] \textit{Branch} per lo sviluppo della parte di \textit{backend} e del \textit{database};
        \item[\texttt{D*} e \texttt{modificheD*}] \textit{Branch} per lo sviluppo dei \textit{deliverable} D* e per le modifiche successive;
        \item[\texttt{UserStory} - \texttt{UserWorkFlow}] \textit{Branch} per la scrittura delle \textit{User Story} il disegno dei \textit{User WorkFlow}. Questi erano \textit{branch} temporanei figli del \textit{branch} \texttt{D2};
        \item[\textit{altri}] Altri \textit{branch} che sono stati creati per lo sviluppo di \textit{pitch} o prime parti iniziali per il \textit{deliverable} \texttt{D1};
    \end{description}
    Come suddivisione dello sviluppo e della scrittura dei \textit{deliverable} si è scelto di assegnare ad ogni membro del gruppo una particolare area di lavoro individuando un responsabile principale per le varie aree di sviluppo e stesura dei documenti, per le slides di \textit{pitch} ognuno ha contribuito in modo equo. Distinguiamo quindi i seguenti ruoli:
    \begin{description}
        \item[Luca Facchini] Responsabile della parte pratica di \textit{backend}, \textit{frontend}. Responsabile principale per la scrittura del documento \texttt{D3}. Addetto alla prima scrittura di requisiti funzionali e non funzionali per il \textit{deliverable} D1; 
        \item[Luca Prigione] Responsabile principale per il documento \texttt{D1} e co-responsabile per il documento \texttt{D2}. Responsabile inoltre per la parte pratica riguardante la struttura del \textit{database}. Addetto alla stesura delle \textit{UserStories} per il \texttt{D3};
        \item[Enrico Faa] Co-responsabile per il documento \texttt{D2} e responsabile di tutti i grafici presenti su tutti i documenti (\texttt{UserWorkFlow}, \texttt{UseCaseDiagram}, \texttt{ClassDiagram}, \dots). Addetto alla stesura delle \textit{UserStories} per il \texttt{D3} ed addetto alla parte di \textit{Testing} per il \texttt{D3}; Addetto inoltre alla prima stesura della descrizione del progetto
    \end{description}
\section{\textit{Dependency}}
    \paragraph{Dipendenze principali} Sono state utilizzate le seguenti dipendenze principali:
    \begin{description}
        \item[\texttt{express}] per la creazione e gestione del \textit{server} e delle rotte;
        \item[\texttt{mongoose}] per la gestione del \textit{database};
        \item[\texttt{mongodb}] per la connessione al \textit{database};
        \item[\texttt{jsonwebtoken}] per la gestione dei \textit{token} di autenticazione;
    \end{description}
    \paragraph{Dipendenze di sviluppo}
    \begin{description}
        \item[\texttt{concurrenly}] per l'esecuzione di più comandi in parallelo durante le fasi di sviluppo e \textit{build} del progetto;
        \item[\texttt{dotenv}] per la gestione delle variabili ambientali;
        \item[\texttt{jest}] per la gestione dei test;
        \item[\texttt{supertest} e \texttt{ts-jest}] per la gestione dei test \textit{end-to-end};
        \item[\texttt{nodemon}] per il \textit{hot-reloading} del \textit{server} in fase di sviluppo;
        \item[\texttt{postcss}] per la gestione dei fogli di stile;
        \item[\texttt{primevue}] per la creazione di alcune componenti grafiche;
        \item[\texttt{tailwindcss}] per la disposizione della \textit{responsiveness} delle componenti;
        \item[\texttt{TypeScript}] per il \textit{type-checking} e la gestione dei tipi;
        \item[\texttt{ts-node}] per l'esecuzione di codice \texttt{TypeScript} direttamente da \texttt{Node.js};
        \item[\texttt{vite}] per la compilazione e il \textit{bundling} del codice;
        \item[\texttt{vue}] per la creazione delle componenti, inoltre i seguenti sotto-moduli sono stati utilizzati:
            \begin{description}
                \item[\texttt{vue-router}] per la gestione delle rotte;
                \item[\texttt{vue-tsc}] per il \textit{type-checking} di \texttt{Vue.js};
                \item[\texttt{vue-leftlet}] per la gestione delle mappe;
            \end{description}
    \end{description}
\section{\textit{Database}}
    Il \textit{database} è stato progettato per contenere le seguenti collezioni:
    \subsection{Quartiere}
        La collezione \texttt{Quartiere} viene usata per memorizzare i dati relativi ai quartieri di Trento. Ogni documento della collezione contiene i seguenti campi:
        \begin{description}
            \item[\texttt{\_id}] (\texttt{ObjectId}) identificativo univoco del quartiere;
            \item[\texttt{nome}] (\texttt{String}) nome del quartiere;
            \item[\texttt{coordinate}] (\texttt{Number[][]}) coordinate del quartiere;
            \item[\texttt{circoscrizione}] (\texttt{ObjectId}) identificativo della circoscrizione a cui appartiene il quartiere;
            \item[\texttt{popolazione}] (\texttt{Number}) numero di abitanti del quartiere;
            \item[\texttt{superficie}] (\texttt{Number}) superficie del quartiere in $km^2$;
            \item[\texttt{serviziTotali}] (\texttt{Number}) numero totale di servizi presenti nel quartiere;
            \item[\texttt{interventiPolizia}] (\texttt{Number}) numero di interventi della polizia nel quartiere;
            \item[\texttt{etaMedia}] (\texttt{Number}) età media degli abitanti del quartiere;
            \item[\texttt{servizi}] (\texttt{Object}) oggetto contenente i servizi presenti nel quartiere;
                \begin{description}
                    \item[\texttt{areeVerdi}] (\texttt{Number}) numero di aree verdi presenti nel quartiere;
                    \item[\texttt{scuole}] (\texttt{Number}) numero di scuole presenti nel quartiere;
                    \item[\texttt{serviziRistorazione}] (\texttt{Number}) numero di servizi di ristorazione presenti nel quartiere;
                    \item[\texttt{localiNotturni}]
                \end{description}
            \item[\texttt{sicurezza}] (\texttt{Object}) oggetto contenente i dati relativi alla sicurezza del quartiere;
                \begin{description}
                    \item[\texttt{numeroInterventi}] (\texttt{Number}) numero di interventi della polizia nel quartiere;
                    \item[\texttt{incidenti}] (\texttt{Number}) numero di incidenti nel quartiere;
                    \item[\texttt{tassoCriminalità}] (\texttt{Number}) tasso di criminalità del quartiere;
                \end{description}
        \end{description}
    \subsection{Circoscrizione}
        La collezione \texttt{Circoscrizione} viene usata solamente per la memorizzazione delle coordinate e del nome delle circoscrizioni di Trento, questo in quanto i dati relativi alle circoscrizioni vengono estrapolati dai dati dei quartieri stessi. Ogni documento della collezione contiene i seguenti campi:
        \begin{description}
            \item[\texttt{\_id}] (\texttt{ObjectId}) identificativo univoco della circoscrizione;
            \item[\texttt{nome}] (\texttt{String}) nome della circoscrizione;
            \item[\texttt{coordinate}] (\texttt{Number[][]}) coordinate della circoscrizione;
        \end{description}
    \subsection{Sondaggio}
        La collezione \texttt{Sondaggio} viene usata per memorizzare i dati relativi ai sondaggi effettuati dai sondaggisti, in particolare 
        Ogni documento della collezione contiene i seguenti campi:
        \begin{description}
            \item[\texttt{\_id}] (\texttt{ObjectId}) identificativo univoco del sondaggio;
            \item[\texttt{titolo}] (\texttt{String}) titolo del sondaggio;
            \item[\texttt{dataInizio}] (\texttt{Date}) data di inizio del sondaggio;
            \item[\texttt{isAperto}] (\texttt{Boolean}) \textit{flag} che indica se il sondaggio è aperto o chiuso;
            \item[\texttt{statoApprovazione}] (\texttt{In attesa|Approvato|Rifiutato}) stato di approvazione del sondaggio;
            \item[\texttt{sondaggista}] (\texttt{ObjectId}) identificativo del sondaggista;
        \end{description}
    \subsection{\textit{User}}
        La collezione \texttt{User} viene usata per memorizzare i dati relativi agli utenti dell'applicazione. Questa collection viene strutturata in questo modo in quanto per lo scopo del progetto non viene implementato il sistema di autenticazione \texttt{SSO} tramite \texttt{SPID}/\texttt{CIE}/\texttt{TS-CNS}. Nel caso si dovesse implementare un sistema di autenticazione tramite \texttt{SSO} pubblico la collezione conterrebbe solo una associazione tra l'utente e i suoi dati personali.
        Ogni documento della collezione contiene i seguenti campi:
        \begin{description}
            \item[\texttt{\_id}] (\texttt{ObjectId}) identificativo univoco dell'utente;
            \item[\texttt{nome}] (\texttt{String}) nome dell'utente;
            \item[\texttt{cognome}] (\texttt{String}) cognome dell'utente;
            \item[\texttt{email}] (\texttt{String}) email dell'utente;
            \item[\texttt{password}] (\texttt{String}) password dell'utente;
            \item[\texttt{ruolo}] (\texttt{Amministratore|Analista|Sondaggista|Circoscrizione}) ruolo dell'utente;
            \item[\texttt{imageUrl}] (\texttt{String}) \textit{Hash} dell'immagine profilo dell'utente;
        \end{description}
    \subsection{Voto}
        La collezione \texttt{Voto} viene usata per memorizzare i voti dati dai cittadini ai quartieri. Ogni documento della collezione contiene i seguenti campi:
        \begin{description}
            \item[\texttt{\_id}] (\texttt{ObjectId}) identificativo univoco del voto;
            \item[\texttt{eta}] (\texttt{Number}) età del votante;
            \item[\texttt{voto}] (\texttt{Number}) voto dato al quartiere;
            \item[\texttt{quartiere}] (\texttt{ObjectId}) identificativo del quartiere votato;
            \item[\texttt{dataOra}] (\texttt{Date}) data e ora del voto;
            \item[\texttt{sondaggio}] (\texttt{ObjectId}) identificativo del sondaggio a cui il voto è associato;
        \end{description}
\section{Testing}

    Per ognuno degli endpoint dell'API implementati nell'applicazione sono stati individuati dei casi di test per assicurare la loro corretta implementazione e, in generale, il corretto funzionamento degli elementi di backend del sistema, anche in situazioni di casi limite.
    L’implementazione dei test è organizzata in file .test.ts locati nella cartella \textit{tests}.

        \footnotesize
        \centering
        \begin{xltabular}{\textwidth}{|l|X|X|X|c|X|X|c|}

            \hline \multicolumn{1}{|l|}{\textbf{N.}} & \multicolumn{1}{X|}{\textbf{Descrizione}} & \multicolumn{1}{X|}{\textbf{\textit{Test Data}}} & \multicolumn{1}{X|}{\textbf{Precondizioni}} & \multicolumn{1}{c|}{\textbf{Dipendenze}} & \multicolumn{1}{X|}{\textbf{Risultato Atteso}} & \multicolumn{1}{X|}{\textbf{Risultato Riscontrato}} & \multicolumn{1}{c|}{\textbf{Note}}\\ \hline 
            \endfirsthead
            
            \multicolumn{8}{l}%
            {Tabella continuata dalla pagina precedente} \\
            \hline \multicolumn{1}{|l|}{\textbf{N.}} & \multicolumn{1}{X|}{\textbf{Descrizione}} & \multicolumn{1}{X|}{\textbf{\textit{Test Data}}} & \multicolumn{1}{X|}{\textbf{Precondizioni}} & \multicolumn{1}{c|}{\textbf{Dipendenze}} & \multicolumn{1}{X|}{\textbf{Risultato Atteso}} & \multicolumn{1}{X|}{\textbf{Risultato Riscontrato}} & \multicolumn{1}{c|}{\textbf{Note}}\\ \hline
            \endhead
            
            \hline \multicolumn{8}{|r|}{{Tabella continuata nella pagina successiva}} \hline
            \endfoot
            
            \hline
            \endlastfoot
        
            \hline
            1.1 & Ottenimento lista quartieri base senza coordinate & \textit{deepData=false}, \textit{coordinate=false} & - & - & Lista quartieri senza coordinate & Lista quartieri senza coordinate & - \\
            \hline
            1.2 & Ottenimento lista quartieri base con coordinate & \textit{deepData=false} \textit{coordinate=true} & - & - & Lista quartieri con coordinate & Lista quartieri con coordinate & - \\
            \hline
            1.3 & Ottenimento lista quartieri dettagliata senza coordinate & \textit{deepData=true} \textit{coordinate=false} & - & - & Lista quartieri dettagliata senza coordinate & Lista quartieri dettagliata senza coordinate & - \\
            \hline
            1.4 & Ottenimento lista quartieri dettagliata con coordinate & \textit{deepData=true} \textit{coordinate=true} & - & - & Lista quartieri dettagliata con coordinate & Lista quartieri dettagliata con coordinate & - \\
            \hline
            2.1 & Ottenimento singolo quartiere con coordinate & \textit{quartiereId}, \textit{coordinate=true} & Il quartiere deve esistere & - & Quartiere con coordinate & Quartiere con coordinate & - \\
            \hline
            2.2 & Ottenimento singolo quartiere senza coordinate & \textit{quartiereId}, \textit{coordinate=false} & Il quartiere deve esistere & - & Quartiere senza coordinate & Quartiere senza coordinate & - \\
            \hline
            2.3 & Ottenimento quartiere non esistente & \textit{quartiereId}, \textit{coordinate=any} & Il quartiere non deve esistere & - & Errore \texttt{404} & Errore \texttt{404} & - \\
            \hline
            2.4 & Ottenimento quartiere con \texttt{ID} non valido & \textit{quartiereId}, \textit{coordinate=any} & Il \texttt{ID} del quartiere non deve essere valido & - & Errore \texttt{400} & Errore \texttt{400} & - \\
            \hline
            3.1 & Ottenimento lista circoscrizioni base senza coordinate & \textit{deepData=false}, \textit{coordinate=false} & - & - & Lista circoscrizioni senza coordinate & Lista circoscrizioni senza coordinate & - \\
            \hline
            3.2 & Ottenimento lista circoscrizioni base con coordinate & \textit{deepData=false} \textit{coordinate=true} & - & - & Lista circoscrizioni con coordinate & Lista circoscrizioni con coordinate & - \\
            \hline
            3.3 & Ottenimento lista circoscrizioni dettagliata senza coordinate & \textit{deepData=true} \textit{coordinate=false} & - & - & Lista circoscrizioni dettagliata senza coordinate & Lista circoscrizioni dettagliata senza coordinate & - \\
            \hline
            3.4 & Ottenimento lista circoscrizioni dettagliata con coordinate & \textit{deepData=true} \textit{coordinate=true} & - & - & Lista circoscrizioni dettagliata con coordinate & Lista circoscrizioni dettagliata con coordinate & - \\
            \hline
            4.1 & Ottenimento singola circoscrizione con coordinate & \textit{circoscrizioneId}, \textit{coordinate=true} & La circoscrizione deve esistere & - & Circoscrizione con coordinate & Circoscrizione con coordinate & - \\
            \hline
            4.2 & Ottenimento singola circoscrizione senza coordinate & \textit{circoscrizioneId}, \textit{coordinate=false} & La circoscrizione deve esistere & - & Circoscrizione senza coordinate & Circoscrizione senza coordinate & - \\
            \hline
            4.3 & Ottenimento circoscrizione non esistente & \textit{circoscrizioneId}, \textit{coordinate=any} & La circoscrizione non deve esistere & - & Errore \texttt{404} & Errore \texttt{404} & - \\
            \hline
            4.4 & Ottenimento circoscrizione con \texttt{ID} non valido & \textit{circoscrizioneId}, \textit{coordinate=any} & Il \texttt{ID} della circoscrizione non deve essere valido & - & Errore \texttt{400} & Errore \texttt{400} & - \\
            \hline
            5 & Ottenimento informazioni generali di Trento & - & - & - & Informazioni generali della città di Trento & Informazioni generali della città di Trento & - \\
            \hline
            6.1 & Ottenimento informazioni di una sessione & \textit{JWT token} & Il token deve essere valido & - & Informazioni dell'utente & Informazioni dell'utente & - \\
            \hline
            6.2 & Ottenimento informazioni di una sessione revocata & \textit{JWT token} & Il token deve essere valido e corrispondere a una sessione che è stata revocata & - & Errore \texttt{401} & Errore \texttt{401} & - \\
            \hline
            6.3 & Ottenimento informazioni di una sessione non valida & \textit{JWT token} & Il token non deve essere valido & - & Errore \texttt{401} & Errore \texttt{401} & - \\
            \hline
            6.4 & Ottenimento informazioni di una sessione con token in format sbagliato & \textit{JWT token} & Il token deve essere in un format sbagliato & - & Errore \texttt{401} & Errore \texttt{401} & - \\
            \hline
            7.1 & Creazione sessione & \textit{nomeUtente}, \textit{password} & Le credenziali devono essere valide & - & JWT token valido, informazioni dell'utente & JWT token valido, informazioni dell'utente & - \\
            \hline
            7.2 & Creazione sessione senza nome utente & \textit{password} & - & - & Errore \texttt{400} & Errore \texttt{400} & - \\
            \hline
            7.3 & Creazione sessione senza password & \textit{nomeUtente} & - & - & Errore \texttt{400} & Errore \texttt{400} & - \\
            \hline
            7.4 & Creazione sessione con password errata & \textit{nomeUtente}, \textit{password} & Il nome utente deve essere valido, ma la password non deve essere quella di quell'utente & - & Errore \texttt{401} & Errore \texttt{401} & - \\
            \hline
            8.1 & Revoca sessione & \textit{JWT token} & Il token deve essere valido & - & La sessione viene revocata & La sessione viene revocata & - \\
            \hline
            8.2 & Revoca sessione senza token & - & - & - & Errore \texttt{401} & Errore \texttt{401} & - \\
            \hline
            8.3 & Revoca sessione con token non valido & \textit{JWT token} & Il token non deve essere valido & - & Errore \texttt{401} & Errore \texttt{401} & - \\
            \hline
            8.4 & Revoca sessione con token un format sbagliato & \textit{JWT token} & Il token deve essere in un format sbagliato & - & Errore \texttt{401} & Errore \texttt{401} & - \\
            \hline
            9.1 & Ottenimento lista sondaggi base sondaggista & \textit{JWT token}, \textit{deepData=false} & Il token deve essere valido e appartenente a un utente sondaggista & - & Lista sondaggi base del sondaggista & Lista sondaggi base del sondaggista & - \\
            \hline
            9.2 & Ottenimento lista sondaggi dettagliata sondaggista & \textit{JWT token}, \textit{deepData=true} & Il token deve essere valido e appartenere a un utente sondaggista & - & Lista dettagliata dei sondaggi del sondaggista & Lista dettagliata dei sondaggi del sondaggista & - \\
            \hline
            9.3 & Ottenimento lista sondaggi base amministratore & \textit{JWT token}, \textit{deepData=false} & Il token deve essere valido e appartenente a un utente amministratore & - & Lista sondaggi base & Lista sondaggi base & - \\
            \hline
            9.4 & Ottenimento lista sondaggi dettagliata amministratore & \textit{JWT token}, \textit{deepData=true} & Il token deve essere valido e appartenere a un utente amministratore & - & Lista dettagliata dei sondaggi & Lista dettagliata dei sondaggi & - \\
            \hline
            9.5 & Ottenimento lista sondaggi senza token & \textit{deepData=any} & - & - & Errore \texttt{401} & Errore \texttt{401} & - \\
            \hline
            9.6 & Ottenimento lista sondaggi con token non valido & \textit{JWT token}, \textit{deepData=any} & Il token non deve essere valido & - & Errore \texttt{401} & Errore \texttt{401} & - \\
            \hline
            9.7 & Ottenimento lista sondaggi senza autorizzazione & \textit{JWT token}, \textit{deepData=any} & Il token deve essere valido ma non appartenente a un utente sondaggista o amministratore & - & Errore \textt{403} & Errore \textt{403} & - \\
            \hline
            10.1 & Creazione sondaggio & \textit{JWT token}, \textit{titoloSondaggio} & Il token deve essere valido e appartenente a un utente sondaggista & - & Creazione sondaggio con titolo dato & Creazione sondaggio con titolo dato & - \\
            \hline
            10.2 & Creazione sondaggio senza titolo & \textit{JWT token} & Il token deve essere valido e appartenente a un utente sondaggista & - & Errore \texttt{400} & Errore \texttt{400} & - \\
            \hline
            10.3 & Creazione sondaggio con titolo vuoto & \textit{JWT token}, \textit{titoloSondaggio} & Il token deve essere valido e appartenente a un utente sondaggista, il titolo deve essere vuoto & - & Errore \texttt{400} & Errore \texttt{400} & - \\
            \hline
            10.4 & Creazione sondaggio senza token & \textit{titoloSondaggio} & - & - & Errore \texttt{401} & Errore \texttt{401} & - \\
            \hline
            10.5 & Creazione sondaggio con token non valido & \textit{JWT token}, \textit{titoloSondaggio} & Il token non deve essere valido & - & Errore \texttt{401} & Errore \textt{401} & - \\
            \hline
            10.6 & Creazione sondaggio senza autorizzazione & \textit{JWT token}, \textit{titoloSondaggio} & il token deve essere valido ma non appartenente a un utente sondaggista & - & Errore \texttt{403} & Errore \texttt{403} & - \\
            \hline
            11.1 & Ottenimento singolo sondaggio sondaggista & \textit{JWT token}, \textit{sondaggioId} & Il token deve essere valido e appartenente al sondaggista proprietario del sondaggio, il sondaggio deve esistere & - & Informazioni sondaggio & Informazioni sondaggio & - \\
            \hline
            11.2 & Ottenimento singolo sondaggio amministratore & \textit{JWT token}, \textit{sondaggioId} & Il token deve essere valido e appartenente a un amministratore, il sondaggio deve esistere & - & Informazioni sondaggio & Informazioni sondaggio & - \\
            \hline
            11.3 & Ottenimento singolo sondaggio non esistente & \textit{JWT token}, \textit{sondaggioId} & Il token deve essere valido e appartenente a un utente sondaggista, il sondaggio non deve esistere & - & Errore \texttt{404} & Errore \texttt{404} & - \\
            \hline
            11.4 & Ottenimento singolo sondaggio con \texttt{ID} non valido & \textit{JWT token}, \textit{sondaggioId} & Il \texttt{ID} del sondaggio non deve essere valido & - & Errore \texttt{400} & Errore \texttt{400} & - \\
            \hline
            11.5 & Ottenimento singolo sondaggio senza token & \textit{sondaggioId} & - & - & Errore \textt{401} & Errore \textt{401} & - \\
            \hline
            11.6 & Ottenimento singolo sondaggio con token non valido & \textit{JWT token}, \textit{sondaggioId} & Il token non deve essere valido & - & Errore \texttt{401} & Errore \texttt{401} & - \\
            \hline
            11.7 & Ottenimento singolo sondaggio senza autorizzazione & \textit{JWT token}, \textit{sondaggioId} & Il token deve essere valido ma non appartenente a un amministratore o al sondaggista proprietario del sondaggio & - & Errore \textt{403} & Errore \textt{403} & - \\
            \hline
            12.1 & Chiusura sondaggio & \textit{JWT token}, \textit{sondaggioId}, \textit{isAperto=false} & Il token deve essere valido e appartenente al sondaggista proprietario del sondaggio, il sondaggio deve esistere, il sondaggio deve essere aperto & - & Chiusura del sondaggio & Chiusura del sondaggio & - \\
            \hline
            12.2 & Chiusura sondaggio non esistente & \textit{JWT token}, \textit{sondaggioId}, \textit{isAperto=any} & Il token deve essere valido e appartenente a un sondaggista, il sondaggio non deve esistere & - & Errore \texttt{404} & Errore \texttt{404} & - \\
            \hline
            12.3 & Chiusura sondaggio con \texttt{ID} non valido & \textit{JWT token}, \textit{sondaggioId}, \textit{isAperto=any} & Il token deve essere valido e appartenente a un sondaggista, il \texttt{ID} del sondaggio non deve essere valido & - & Errore \texttt{400} & Errore \texttt{400} & - \\
            \hline
            12.4 & Chiusura sondaggio senza token & \textit{sondaggioId}, \textit{isAperto=any} & - & - & Errore \texttt{401} & Errore \texttt{401} & - \\
            \hline
            12.5 & Chiusura sondaggio con token non valido & \textit{JWT token}, \textit{sondaggioId}, \textit{isAperto=any} & Il token non deve essere valido & - & Errore \texttt{401} & Errore \texttt{401} & - \\
            \hline
            12.6 & Chiusura sondaggio senza autorizzazione & \textit{JWT token}, \textit{sondaggioId}, \textit{isAperto=any} & Il token deve essere valido ma non appartenente al sondaggista proprietario del sondaggio & - & Errore \texttt{403} & Errore \texttt{403} & - \\
            \hline
            12.7 & Chiusura sondaggio senza body & \textit{JWT token}, \textit{sondaggioId}, & Il token deve essere valido e appartenente al sondaggista proprietario del sondaggio, il sondaggio deve esistere & - & Errore \texttt{400} & Errore \texttt{400} & - \\
            \hline
            12.8 & Chiusura sondaggio già chiuso & \textit{JWT token}, \textit{sondaggioId}, \textit{isAperto=false} & Il token deve essere valido e appartenente al sondaggista proprietario del sondaggio, il sondaggio deve esistere, il sondaggio deve essere chiuso & - & Errore \texttt{403} & Errore \texttt{403} & - \\
            \hline
            13.1 & Eliminazione sondaggio & \textit{JWT token}, \textit{sondaggioId} & Il token deve essere valido e appartenente al sondaggista proprietario del sondaggio, il sondaggio deve esistere, il sondaggio deve essere aperto & - & Eliminazione del sondaggio & Eliminazione del sondaggio & - \\
            \hline
            13.2 & Eliminazione sondaggio rifiutato amministratore & \textit{JWT token}, \textit{sondaggioId} & Il token deve essere valido e appartenente a un amministratore, il sondaggio deve esistere, il sondaggio deve essere chiuso e rifiutato & - & Eliminazione del sondaggio & Eliminazione del sondaggio & - \\
            \hline
            13.3 & Eliminazione sondaggio non esistente & \textit{JWT token}, \textit{sondaggioId} & Il token deve essere valido e appartenente a un sondaggista o un amministratore, il sondaggio non deve esistere & - & Errore \texttt{404} & Errore \texttt{404} & - \\
            \hline
            13.4 & Eliminazione sondaggio con \texttt{ID} non valido & \textit{JWT token}, \textit{sondaggioId} & Il token deve essere valido e appartenente un sondaggista o un amministratore, il \texttt{ID} del sondaggio non deve essere valido & - & Errore \texttt{400} & Errore \texttt{400} & - \\
            \hline
            13.5 & Eliminazione sondaggio senza token & \textit{sondaggioId} & - & - & Errore \texttt{401} & Errore \texttt{401} & - \\
            \hline
            13.6 & Eliminazione sondaggio con token non valido & \textit{JWT token}, \textit{sondaggioId} & Il token non deve essere valido & - & Errore \texttt{401} & Errore \texttt{401} & - \\
            \hline
            13.7 & Eliminazione sondaggio senza autorizzazione & \textit{JWT token}, \textit{sondaggioId} & Il token deve essere valido ma non appartenente al sondaggista proprietario del sondaggio o un amministratore & - & Errore \texttt{403} & Errore \texttt{403} & - \\
            \hline
            13.8 & Eliminazione sondaggio chiuso sondaggista & \textit{JWT token}, \textit{sondaggioId} & Il token deve essere valido e appartenente al sondaggista proprietario del sondaggio, il sondaggio deve esistere, il sondaggio deve essere chiuso & - & Errore \texttt{403} & Errore \texttt{403} & - \\
            \hline
            13.9 & Eliminazione sondaggio aperto amministratore & \textit{JWT token}, \textit{sondaggioId} & Il token deve essere valido e appartenente a un amministratore, il sondaggio deve esistere, il sondaggio deve essere aperto & - & Errore \texttt{403} & Errore \texttt{403} & - \\
            \hline
            13.10 & Eliminazione sondaggio accettato amministratore & \textit{JWT token}, \textit{sondaggioId} & Il token deve essere valido e appartenente a un amministratore, il sondaggio deve esistere, il sondaggio deve essere chiuso e accettato & - & Errore \texttt{403} & Errore \texttt{403} & - \\
            \hline
            14.1 & Ottenimento lista voti & \textit{JWT token}, \textit{sondaggioId} & Il token deve essere valido e appartenente al sondaggista proprietario del sondaggio, il sondaggio deve esistere & - & Lista voti del sondaggio & Lista voti del sondaggio & - \\
            \hline
            14.2 & Ottenimento lista voti di sondaggio non esistente & \textit{JWT token}, \textit{sondaggioId} & Il token deve essere valido e appartenente a un sondaggista, il sondaggio non deve esistere & - & Errore \texttt{404} & Errore \texttt{404} & - \\
            \hline
            14.3 & Ottenimento lista voti con \texttt{ID} sondaggio non valido & \textit{JWT token}, \textit{sondaggioId} & Il token deve essere valido e appartenente a un sondaggista, il \texttt{ID} del sondaggio non deve essere valido & - & Errore \texttt{400} & Errore \texttt{400} & - \\
            \hline
            14.4 & Ottenimento lista voti senza token & \textit{JWT token}, \textit{sondaggioId} & - & - & Errore \texttt{401} & Errore \texttt{401} & - \\
            \hline
            14.5 & Ottenimento lista voti con token non valido & \textit{JWT token}, \textit{sondaggioId} & Il token non deve essere valido & - & Errore \texttt{401} & Errore \texttt{401} & - \\
            \hline
            14.6 & Ottenimento lista voti senza autorizzazione & \textit{JWT token}, \textit{sondaggioId} & Il token deve essere valido ma non appartenente al sondaggista proprietario del sondaggio & - & Errore \texttt{403} & Errore \texttt{403} & - \\
            \hline
            15.1 & Aggiunta voto & \textit{JWT token}, \textit{sondaggioId}, \textit{eta}, \textit{voto}, \textit{quartiere} & Il token deve essere valido e appartenente al sondaggista proprietario del sondaggio, il sondaggio deve esistere, il quartiere deve esistere & - & Creazione voto nel sondaggio & Creazione voto nel sondaggio & - \\
            \hline
            15.2 & Aggiunta voto senza età & \textit{JWT token}, \textit{sondaggioId}, \textit{eta=null}, \textit{voto}, \textit{quartiere} & Il token deve essere valido e appartenente al sondaggista proprietario del sondaggio, il sondaggio deve esistere, il quartiere deve esistere & - & Creazione voto nel sondaggio & Creazione voto nel sondaggio & - \\
            \hline
            15.3 & Aggiunta voto con \texttt{eta} non valida & \textit{JWT token}, \textit{sondaggioId}, \textit{eta}, \textit{voto}, \textit{quartiere} & Il token deve essere valido e appartenente al sondaggista proprietario del sondaggio, il sondaggio deve esistere, \textit{eta}<0 oppure \textit{eta}>100 & - & Errore \texttt{400} & Errore \texttt{400} & - \\
            \hline
            15.4 & Aggiunta voto a sondaggio senza valore voto & \textit{JWT token}, \textit{sondaggioId}, \textit{eta}, \textit{quartiere} & Il token deve essere valido e appartenente al sondaggista proprietario del sondaggio, il sondaggio deve esistere & - & Errore \texttt{400} & Errore \texttt{400} & - \\
            \hline
            15.5 & Aggiunta voto con valore \textit{voto} non valido & \textit{JWT token}, \textit{sondaggioId}, \textit{eta}, \textit{voto}, \textit{quartiere} & Il token deve essere valido e appartenente al sondaggista proprietario del sondaggio, il sondaggio deve esistere, \textit{voto}<1 oppure \textit{voto}>5 & - & Errore \texttt{400} & Errore \texttt{400} & - \\
            \hline
            15.6 & Aggiunta voto senza quartiere & \textit{JWT token}, \textit{sondaggioId}, \textit{eta}, \textit{voto} & Il token deve essere valido e appartenente al sondaggista proprietario del sondaggio, il sondaggio deve esistere & - & Errore \texttt{400} & Errore \texttt{400} & - \\
            \hline
            15.7 & Aggiunta voto con quartiere non esistente & \textit{JWT token}, \textit{sondaggioId}, \textit{eta}, \textit{voto}, \textit{quartiere} & Il token deve essere valido e appartenente al sondaggista proprietario del sondaggio, il sondaggio deve esistere, il quartiere non deve esistere & - & Errore \texttt{400} & Errore \texttt{400} & - \\
            \hline
            15.8 & Aggiunta voto vuoto & \textit{JWT token}, \textit{sondaggioId} & Il token deve essere valido e appartenente al sondaggista proprietario del sondaggio, il sondaggio deve esistere & - & Errore \texttt{400} & Errore \texttt{400} & - \\
            \hline
            15.9 & Aggiunta voto con \texttt{ID} sondaggio non valido & \textit{JWT token}, \textit{sondaggioId}, \textit{eta}, \textit{voto}, \textit{quartiere} & Il token deve essere valido e appartenente a un sondaggista, il \texttt{ID} del sondaggio non deve essere valido & - & Errore \texttt{40} & Errore \texttt{40} & - \\
            \hline
            15.10 & Aggiunta voto in sondaggio non esistente & \textit{JWT token}, \textit{sondaggioId}, \textit{eta}, \textit{voto}, \textit{quartiere} & Il token deve essere valido e appartenente al sondaggista proprietario del sondaggio, il sondaggio non deve esistere & - & Errore \texttt{404} & Errore \texttt{404} & - \\
            \hline
            15.11 & Aggiunta voto senza token & \textit{sondaggioId}, \textit{eta}, \textit{voto}, \textit{quartiere} & - & - & Errore \texttt{401} & Errore \texttt{401} & - \\
            \hline
            15.12 & Aggiunta voto con token non valido & \textit{JWT token}, \textit{sondaggioId}, \textit{eta}, \textit{voto}, \textit{quartiere} & Il token non deve essere valido & - & Errore \texttt{401} & Errore \texttt{401} & - \\
            \hline
            15.13 & Aggiunta voto senza autorizzazione & \textit{JWT token}, \textit{sondaggioId}, \textit{eta}, \textit{voto}, \textit{quartiere} & Il token deve essere valido ma non appartenente al sondaggista proprietario del sondaggio & - & Errore \texttt{403} & Errore \texttt{403} & - \\
            \hline
            16.1 & Eliminazione voto & \textit{JWT token}, \textit{votoId} & Il token deve essere valido e appartenente al sondaggista proprietario del sondaggio che contiene il voto, il voto deve esistere & - & Eliminazione voto dal sondaggio & Eliminazione voto dal sondaggio & - \\
            \hline
            16.1 & Eliminazione voto non esistente & \textit{JWT token}, \textit{votoId} & Il token deve essere valido e appartenente a un sondaggista, il voto non deve esistere & - & Errore \texttt{404} & Errore \texttt{404} & - \\
            \hline
            16.1 & Eliminazione voto con \texttt{ID} non valido & \textit{JWT token}, \textit{votoId} & Il token deve essere valido e appartenente al sondaggista proprietario del sondaggio che contiene il voto, il \texttt{ID} del voto non deve essere valido & - & Errore \texttt{400} & Errore \texttt{400} & - \\
            \hline
            16.1 & Eliminazione voto senza token & \textit{JWT token}, \textit{votoId} & - & - & Errore \texttt{401} & Errore \texttt{401} & - \\
            \hline
            16.1 & Eliminazione voto con token non valido & \textit{JWT token}, \textit{votoId} & Il token non deve essere valido & - & Errore \texttt{401} & Errore \texttt{401} & - \\
            \hline
            16.1 & Eliminazione voto senza autorizzazione & \textit{JWT token}, \textit{votoId} & Il token deve essere valido ma non appartenente al sondaggista proprietario del sondaggio che contiene il voto & - & Errore \texttt{403} & Errore \texttt{403} & - \\
            \hline
        \end{xltabular}

