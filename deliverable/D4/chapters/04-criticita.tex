\chapter{Criticità}
Durante il progetto, come già precedentemente descritto, si è passati da una suddivisione equa per tutte le parti del progetto ad una suddivisione sulla base delle competenze pregresse. Questo ha portato ad un aumento del carico di lavoro per \textit{Prigione Luca} e \textit{Faa Enrico} nella prima parte, questo a discapito di \textit{Facchini Luca} che ha avuto un carico di lavoro minore nella fase di \texttt{D2} ma un maggiore carico di lavoro nella fase di \texttt{D3}. Questo squilibrio è stato dovuto al fatto che \textit{Facchini Luca}. Inoltre mentre durante le lezioni si è riuscito a lavorare con costanza e a mantenere un ritmo di lavoro costante, durante le vacanze natalizie ed la sessione di gennaio il ritmo di lavoro è calato drasticamente, questo a causa della concorrente preparazione degli esami. Questo ha portato ad un aumento del carico di lavoro nelle settimane successive per recuperare il ritardo accumulato.\newline
Una importate criticità si è mostrata quando si è passati alla stesura del \texttt{D2} e lo sviluppo del codice, si è infatti riscontrata una grande discrepanza tra quello che è stato scritto nel documento \texttt{D1} e quello che era stato pensato o sviluppato. Questo ha portato ad una revisione totale del documento \texttt{D1} e ad un rallentamento del lavoro. \newline
Anche se le componenti del gruppo non hanno lavorato tutte equamente allo stesso modo in tutti i \textit{deliverable}, il lavoro è stato comunque equamente distribuito e tutti i membri del gruppo hanno contribuito in modo significativo ed equo al completamento del progetto, inoltre chi non ha preso parte attiva alla stesura di un \textit{deliverable} ha comunque contribuito alla revisione e alla correzione dello stesso, difatti ogni componente ha revisionato almeno una volta ogni \textit{deliverable} al quale \underline{non} ha preso parte attiva alla stesura, ciò per garantire la qualità del lavoro svolto.