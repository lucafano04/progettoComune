\chapter{Distribuzione e suddivisione del carico di lavoro}
    Come precedentemente anticipato ogni membro ha contribuito in qualche modo al completamente di tutti i \textit{deliverable}. Di seguito verrà descritto il carico di lavoro in numero di ore per ciascun membro del gruppo suddiviso per \textit{deliverable}.
    \begin{table}[H]
        \centering
        \begin{tabular}{|c|c|c|c|c|c|}
            \hline
            & \texttt{D1} & \texttt{D2} & \texttt{D3} & \texttt{D4} & \textbf{Totale}\\
            \hline
            \textbf{Facchini Luca} & 18 & 18 & 77 & 10 & 123\\
            \hline
            \textbf{Prigione Luca} & 45 & 67 & 16 & 3 & 131\\
            \hline
            \textbf{Faa Enrico} & 37 & 43 & 34 & 3 & 117\\
            \hline
            \textbf{Totale} & 100 & 128 & 127 & 16 & 371\\
            \hline
        \end{tabular}
    \end{table}
    Alcuni squilibri che si possono notre soprattutto nei \textit{deliverable} \texttt{D2} e \texttt{D3} sono dovuti al fatto che le più avanzate competenze precedentemente acquisite da \textit{Facchini Luca} in materia di sviluppo web e di \textit{Prigione Luca} in materia di progettazione hanno permesso di completare più velocemente la parte di sviluppo e progettazione rispettivamente. Il carico di lavoro di \textit{Faa Enrico} è stato più equilibrato rispetto agli altri due membri del gruppo in quanto la sua attività orizzontale ha si richiesto un impegno costante in tutti i \textit{deliverable}, infatti per il \texttt{D1} e per il \texttt{D2} ha disegnato ed aiutato a definire tutti i vari diagrammi ed interfacce grafiche, mentre per il \texttt{D3} ha progettato l'interfaccia grafica ed eseguito il \textit{testing} del progetto.